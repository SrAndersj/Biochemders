\documentclass[10pt,a4paper]{article}

\usepackage{chemfig}

\begin{document}
	
	
	\chemfig{*6(--=---)}
	
	\vspace{1cm}
	
	\chemfig{H-[1]O-[7]H}
		
	\vspace{1cm}
	
	\chemfig{C-[1]1}
			
	\vspace{1cm}
	
	\chemfig{C-[2]2}
	
		\vspace{1cm}
	
	\chemfig{C-[3]3}
	
		\vspace{1cm}
	
	\chemfig{C-[4]4}
	
		\vspace{1cm}
	
	\chemfig{A~B}
	
		\vspace{1cm}
		
		\chemfig{A>B}
		
			\vspace{1cm}
		
		\chemfig{A<B}
			\vspace{1cm}
			
			
		\chemfig{C-[1]C-[7]C(-[6]H)-[1]C}
		
		
		\chemfig{*5(-=-O-(-(-[5]H)=[2]O)-)}
		
		
		\vspace{1cm}
		
		\chemfig{
			% 7
			-[:210]% 3
			=_[:270]% 2
			-[:210]% 1
			=_[:150]% 6
			-[:90]% 5
			=_[:30]% 4
			(
			-[:330]% -> 3
			)
		}
	
	
	
		\vspace{1cm}
		
		
	\chemfig{
		F% 8
		-[:150]% 7
		(
		-[:90]F% 9
		)
		(
		-[:30]F% 10
		)
		-[:210]% 3
		=_[:270]% 2
		-[:210]% 1
		=_[:150]% 6
		-[:90]% 5
		=_[:30]% 4
		(
		-[:330]% -> 3
		)
	}
		
	
\end{document}